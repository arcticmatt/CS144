%%%%%%%%%%%%%%%%%%%%%%%%%%%%%% Preamble
\documentclass{article}
\usepackage{amsmath,amssymb,amsthm,fullpage}
\usepackage[a4paper,bindingoffset=0in,left=1in,right=1in,top=1in,
bottom=1in,footskip=0in]{geometry}
\newtheorem*{prop}{Proposition}
%\newcounter{Examplecount}
%\setcounter{Examplecount}{0}
\newenvironment{discussion}{\noindent Discussion.}{}
\setlength{\headheight}{12pt}
\setlength{\headsep}{10pt}
\usepackage{fancyhdr}
\pagestyle{fancy}
\fancyhf{}
\lhead{CS144 Eval 1}
\rhead{Matt Lim}
\pagenumbering{gobble}
\usepackage{graphicx}
\usepackage{bbm}
\graphicspath{ {.} }
\begin{document}

%%%%%%%%%%%%%%%%%%%%%%%%%%%%%% Problem 1
\section*{Problem 1}
\subsection*{(a)}
\subsubsection*{(i)}
Correct.

\subsubsection*{(ii)}
Corect, slightly different approach (use indicator variables instead
of Markov's inequality).

\subsubsection*{(iii)}
Correct.

\subsection*{(b)}
Correct, although different ways of explaining why average distance goes to 1.

%%%%%%%%%%%%%%%%%%%%%%%%%%%%%% Problem 2
\section*{Problem 2}
\subsection*{(a)}
Correct.

\subsection*{(b)}
Correct, different approach in solving.

\subsection*{(c)}
\subsubsection*{(i)}
Correct, different way of thinking about it.

\subsubsection*{(ii)}
Correct.

%%%%%%%%%%%%%%%%%%%%%%%%%%%%%% Problem 3
\section*{Problem 3}
\subsection*{(a)}
Correct.

%%%%%%%%%%%%%%%%%%%%%%%%%%%%%% Problem 4
\section*{Problem 4}
\subsection*{(a)}
Correct.

\subsection*{(c)}
Correct.

\end{document}
